\section{Limitations of \TeX{} in User Experience Design}
\label{sec:ux}

\TeX{} and its derivative, the LaTeX ecosystem, have long occupied a central position in the field of academic typesetting. However, the overall user experience has failed to evolve in sync with the advancement of computing environments and user expectations. From the continuous bloating of distribution sizes to the performance and interaction barriers imposed by the compilation model, and further to the long-standing engineering defects in language design, the TeX ecosystem falls short in modern writing contexts---particularly regarding "usability," "maintainability," and "adaptability to modern workflows." This section analyzes the limitations of TeX's user experience design from multiple dimensions, including deployment costs, language structure, and practical user experience.

\subsection{Issues with Distribution Scale and Deployment Models}

\begin{figure}[htbp]
\centering
\includegraphics[width=0.8\textwidth]{figure/iso_size.pdf}
\caption{TeX Live ISO Size Trends}
\label{fig:texlive-size}
\end{figure}

Figure~\ref{fig:texlive-size} shows the growth of TeX Live over time. Figure~\ref{fig:texlive-installer} illustrates the complex installation interface. In contrast, Mogan STEM uses an on-demand plugin loading mechanism, where features are only loaded when needed. This significantly reduces the installation size and startup time.

\begin{figure}[htbp]
\centering
\includegraphics[width=0.8\textwidth]{figure/texlive-installer.png}
\caption{TeX Live Installer}
\label{fig:texlive-installer}
\end{figure}

\subsection{Real-World Performance Dilemma}

The performance of LaTeX compilation degrades significantly with document complexity. Documents with extensive cross-references, bibliographies, or complex mathematical content require multiple passes and can take minutes to compile. This creates a frustrating edit-compile-view cycle that interrupts the writing flow.

\subsection{Intrinsic Weakness: Absence of Engineering Standards}

LaTeX lacks formal engineering standards for package development. While the \texttt{lppl} license provides some guidance, there are no enforced standards for:

\begin{itemize}
    \item Package documentation quality
    \item API stability and versioning
    \item Error handling and reporting
    \item Testing and validation
\end{itemize}

This has led to a fragmented ecosystem where package quality varies widely, and compatibility issues are common \cite{interview2021}.

\begin{table}[htbp]
\centering
\caption{Core Structural Contradictions in \LaTeX2e\ Language Design and their Engineering Consequences}
\label{tab:latex-contradictions}
\begin{tabular}{p{3.5cm}p{5cm}p{4cm}}
\toprule
\textbf{Contradiction} & \textbf{Description} & \textbf{Impact} \\
\midrule
Global Naming vs.\ Modularity & \LaTeX2e\ lacks namespaces; all commands and variables share a global symbol table & High risk of package conflicts; loading order directly affects behavior; system fragility \\
Text Substitution vs.\ Structured Interfaces & Macros are essentially untyped token substitutions, lacking parameter signatures and type systems & Static checking is difficult; function interfaces are uncomposable; parameter passing is error-prone \\
Local State via Grouping vs.\ Compiler Awareness & State management relies on TeX's grouping and rollback mechanism rather than explicit variable scoping & Compilers cannot resolve scope; state leakage and logical errors are difficult to detect \\
Conventional Interfaces vs.\ Automatic Verification & Interface constraints rely on documentation and voluntary compliance, not language-level enforcement & Parameter types/counts cannot be statically verified; errors are only exposed at runtime \\
Compatibility vs.\ Modern Features & To maintain compatibility with \LaTeX2e, implementation of new features is cumbersome & Requires reliance on advanced packages like \texttt{xparse} or complex hacks; high learning and maintenance costs \\
\bottomrule
\end{tabular}
\end{table}

\subsection{User Experience Barriers from a Practical Perspective}

For users, the LaTeX experience is characterized by:

\begin{enumerate}
    \item \textbf{Steep Learning Curve:} Understanding macro expansion, package interactions, and compilation workflows requires significant investment.
    \item \textbf{Cryptic Error Messages:} Error messages often require deep knowledge of TeX internals to interpret.
    \item \textbf{Fragile Workflows:} Small changes can break the compilation in unexpected ways.
    \item \textbf{Limited Tooling:} Compared to modern IDEs, LaTeX editors offer limited refactoring, navigation, and debugging capabilities.
\end{enumerate}

\subsection{Contributions and Constraints of \LaTeX3}

LaTeX3 represents an effort to modernize the LaTeX codebase by introducing a more structured programming layer (\texttt{expl3}). While these improvements address some of the technical debt accumulated over decades, they remain constrained by the need for backward compatibility. The improvements are incremental rather than fundamental, and users must still engage with the legacy macro system for many tasks.

\subsection{Limitations of Modern Collaboration Platforms: A Case Study of Overleaf}

In response to LaTeX's structural deficiencies regarding collaboration support and usability, the academic community has advanced a series of improvement measures. Among these, online collaboration platforms, exemplified by Overleaf, have significantly streamlined the user experience and reduced the technical barrier to entry \cite{overleaf_docs}.

\subsubsection{Key Improvements by Overleaf}

Overleaf has addressed some of LaTeX's usability issues by providing:

\begin{itemize}
    \item Real-time collaboration features
    \item Simplified compilation management
    \item Version control integration
    \item Template libraries for common document types
\end{itemize}

\subsubsection{Constraints of Underlying Architecture}

However, Overleaf cannot overcome the fundamental limitations of LaTeX's architecture:

\begin{itemize}
    \item Compilation is still batch-oriented and slow
    \item Error messages remain cryptic
    \item Real-time preview requires frequent recompilation
    \item Collaboration is at the text level, not the semantic level
\end{itemize}

\subsubsection{Impact on Trajectory of Technological Evolution}

By providing a veneer of usability over LaTeX's fundamental limitations, platforms like Overleaf may have inadvertently slowed the adoption of genuinely superior alternatives. Users who might otherwise seek better solutions are satisfied with incremental improvements to the existing system.
