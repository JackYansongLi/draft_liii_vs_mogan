\section{An Introduction to \TeX{} Compilation Principles}

\begin{lstlisting}[caption={Minimal example of LaTeX code},label={lst:minimal}]
\documentclass{article}
\usepackage{siunitx}
\begin{document}
Hello, World! from \LaTeX.
The speed of light is \SI{299792458}{\meter\per\second}.
\end{document}
\end{lstlisting}

Listing~\ref{lst:minimal} shows a minimal example of LaTeX code. The source code begins with the \verb|\documentclass| command. Lamport's LaTeX framework simplifies the use of TeX~\cite{lamport1994document} by defining the document class used for the file. Immediately following this, we can use \verb|\usepackage| to import packages. We then use \verb|\begin{document}| and \verb|\end{document}| to mark the start and end of the body content, placing the text between them.

The area between \verb|\documentclass| and \verb|\begin{document}| is called the \textit{preamble}. In addition to importing packages via \verb|\usepackage|, \textit{preamble} is used for global macro definitions and document configuration, which can also be left empty.

Finally, we can compile the code. Running \verb|pdflatex example.tex| generates the \verb|example.pdf| file. Simultaneously, several auxiliary files are generated in the same directory. For instance, the \verb|.aux| file saves intermediate information such as cross-references, tables of contents, and numbering; consequently, documents often require multiple compilations to yield correct results. The \verb|.log| file records the complete compilation process and is the primary resource for locating warnings and errors. The \verb|.dvi| file stands for \textit{DeVice Independent file}, an intermediate format unrelated to the specific output device.

Beyond the \verb|.tex| source files we write, every package and document class consists of files with specific extensions. The following files also frequently appear in LaTeX templates:

\begin{table}[htbp]
\centering
\caption{\LaTeX{} file extensions and their descriptions}
\label{tab:latex-ext}
\begin{tabular}{lll}
\toprule
\textbf{File Extension} & \textbf{File Type} & \textbf{Description} \\
\midrule
\texttt{.sty} & Package file & The package name corresponds to the filename without the extension. \\
\texttt{.cls} & Document class file & The document class name corresponds to the filename without the extension. \\
\texttt{.bib} & BibTeX database file & A database file storing bibliographic information. \\
\texttt{.bst} & BibTeX style file & A template file defining the formatting style for bibliographies. \\
\bottomrule
\end{tabular}
\end{table}

\LaTeX{} generates numerous auxiliary files and logs during the compilation process. Features such as cross-references, bibliographies, tables of contents, and indices require an initial compilation to generate auxiliary files, followed by a subsequent compilation to read these files and produce the correct result. Therefore, complex \LaTeX{} source code often requires multiple compilation passes:

\begin{table}[htbp]
\centering
\caption{\LaTeX{} auxiliary file extensions, tools, and descriptions}
\label{tab:latex-aux}
\begin{tabular}{lll}
\toprule
\textbf{Extension} & \textbf{Tool} & \textbf{Description} \\
\midrule
\texttt{.log} & Typesetting Engine & Records the compilation process \\
\texttt{.aux} & \texttt{latex} & Main auxiliary file for cross-references, TOC, citations \\
\texttt{.toc} & \texttt{latex} & Table of contents file \\
\texttt{.lof} & \texttt{latex} & List of figures file \\
\texttt{.lot} & \texttt{latex} & List of tables file \\
\texttt{.bbl} & \texttt{bibtex} & Bibliography file \\
\texttt{.blg} & \texttt{bibtex} & BibTeX log file \\
\texttt{.idx} & \texttt{latex} & Raw index file for \texttt{makeindex} \\
\texttt{.ind} & \texttt{makeindex} & Formatted index file \\
\texttt{.ilg} & \texttt{makeindex} & \texttt{makeindex} log file \\
\texttt{.out} & \texttt{hyperref} & PDF bookmarks file \\
\bottomrule
\end{tabular}
\end{table}

To intuitively understand why \LaTeX{} documents often require multiple compilations, consider the following minimal example containing cross-references and a table of contents:

\begin{lstlisting}[caption={Cross-reference example},label={lst:crossref}]
\documentclass{article}
\usepackage{hyperref}

\begin{document}
\tableofcontents

\section{Introduction}\label{sec:intro}
See Section~\ref{sec:intro}.
\end{document}
\end{lstlisting}

When compiling this code with \verb|pdflatex| for the first time, \LaTeX{} cannot yet determine section numbers or the content of the table of contents (TOC). The compiler records this information into auxiliary files during the typesetting process: section titles are written to the \verb|.toc| file, and numbering information for \verb|\label{sec:intro}| is written to the \verb|.aux| file. Meanwhile, the \verb|\ref{sec:intro}| in the body text is temporarily output as a placeholder. Consequently, the PDF generated from the first pass has an empty TOC, and cross-references appear as "??".

During the second pass, \LaTeX{} reads the \verb|.aux| and \verb|.toc| files generated previously, obtaining complete numbering and directory information to typeset them correctly into the document. Thus, the cross-references appear correctly, and the TOC is populated. This process demonstrates that the \LaTeX{} typesetting process is essentially an iterative workflow that relies on intermediate results to converge. Any feature involving global information or forward/backward dependencies almost inevitably requires multiple compilation passes to produce the final correct output.

In practice, the typesetting result is determined by the underlying engine. Common engines include:

\begin{itemize}
    \item \textbf{pdfLaTeX}: The most traditional and compatible engine. It generates PDFs directly but has limited support for Unicode and system fonts, relying mostly on TeX's own font system.
    \item \textbf{XeLaTeX}: Renowned for excellent Unicode support and the ability to call operating system fonts directly. It is highly suitable for multi-language typesetting (such as CJK), though its compilation speed and package compatibility is often more complex than pdfLaTeX.
    \item \textbf{LuaLaTeX}: Builds upon XeLaTeX by introducing Lua scripting as a programmable extension layer. This mechanism allows for dynamic customization of typesetting logic, offering the most power but also the highest complexity, with a strong dependency on the quality of templates and packages.
\end{itemize}

