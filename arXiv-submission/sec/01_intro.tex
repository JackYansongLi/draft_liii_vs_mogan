\section{A Brief History of \TeX}

Numerous derivatives of TeX have been emerged in past decades, with LaTeX being the most prominent. Built as a macro language on top of TeX, LaTeX significantly simplifies its usage, allowing users to leverage TeX's powerful typesetting capabilities without needing a deep understanding of intricate commands. By defining commands and templates that align with standard typesetting practices, LaTeX has made the production of scientific literature and books far more efficient and accessible, eventually becoming the de facto standard for scientific document preparation.

In the academic field, the TeX system and LaTeX in particular have become the standard of the scientific community thanks to its exceptional mathematical typesetting capabilities. The core design of TeX originates from the pioneering work of Donald Knuth \citep{knuth1984texbook}. The American Mathematical Society (AMS) strongly encourages mathematicians to submit manuscripts using TeX, and widespread adoption by world-class publishers such as Wesley and IEEE has made it a staple for books and journals. Consequently, TeX occupies a pivotal position in the production of academic papers and monographs, serving as a vital tool for scholarly communication and knowledge dissemination.

However, TeX's original design and its subsequent development trajectory have resulted in numerous legacy issues. This article primarily examines the underlying architecture of TeX and explains why certain design choices have led to significant problems.

On another note, it was long believed that ``What You See Is What You Get'' (WYSIWYG) was incompatible with structured editing. The emergence of TeXmacs, however, proved this assumption fundamentally incorrect. Professor Joris from \'Ecole Polytechnique wrote a critique on this subject (see Joris et al.~\citep{van_der_hoeven_gnu_2001, liiistem2025}). The design philosophy of LaTeX has inspired a series of similar editing software; beyond the aforementioned TeXmacs, these include LyX and the recently popular Typst.

Notably, the formula rendering in Typst and TeXmacs is completely independent of the TeX system, whereas LyX serves as a front-end for TeX. While this article contains significant criticism of the TeX system, we wish to clarify our stance: we are by no means denying TeX's historical status, nor do we suggest it was outdated for its time. We simply argue that today---especially in an era of rapidly advancing AI tools---TeX's underlying design presents many issues.
